\documentclass[floatsintext, colorlinks=true, linkcolor=blue, citecolor=blue, urlcolor=blue]{article}
\usepackage[utf8]{inputenc}
\usepackage{longtable}
\usepackage{tabularx}
% \usepackage[landscape]{geometry} % to have document in landscape
\usepackage[left=2.0cm,right=2.5cm,top=2.5cm,bottom=2.5cm]{geometry} % to control document margins
\usepackage{array}
\usepackage{makecell}
\usepackage{csquotes}
\usepackage{hyperref} % to embed links (incl. citations)
\usepackage[style=apa,sortcites=true,sorting=nyt,backend=biber]{biblatex}
\DeclareLanguageMapping{american}{american-apa}
\addbibresource{example.bib}

\begin{document}
\begin{longtable}{| p{7cm} | p{10cm} |} % You can add as many columns as you need; be sure to adjust their width to fit the page
% the vertical lines '|' indicate the visibility of the vertical lines around your cells.
% You can delete the vertical lines if you want a more APA-conforming layout.

\caption{Literature review..\label{long1}}\\ % always use the double backslash to indicate the end of a row

\hline % with '\hline' you draw a horizontal line
% this defines the column names
\textbf{Citation} & \textbf{Summary}\\ % use the '&' to indicate the next cell within the row
\hline
\endfirsthead

\hline
% If the column spans more than a page, this is what we will see
\multicolumn{2}{|c|}{Continuation of literature review \ref{long1}}\\
\hline
\textbf{Citation} & \textbf{Summary}\\
\hline
\endhead

\hline
\endfoot

\hline
\multicolumn{2}{| c |}{End of Table}\\
\hline\hline
\endlastfoot

% This is where the actual content of the table starts
\textcite{freyPsychologicalDriversIndividual2021} & This is a central paper because... \\
\hline

\textcite{steinerRepresentativeDesignPsychological} & This is another interesting paper because... \\

\hline

\end{longtable}


\end{document}
